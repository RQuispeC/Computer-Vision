\section{Gaussian Pyramid}

The implementation of the Gaussian pyramid follows the next structure:

\begin{lstlisting}[language=python][h!]
class gaussian_pyramid:
    pyramid = []
    
    # Initialize the pyramid based on an image, the number of 
    # level and a kernel (Gaussian kernel is the default)
    
    def __init__(self, img, levels, kernel = None, gauss_kernel_par= 0.3):
    
    # interpolates the missing information
    
    def interpolation(self,x,y,v, interp = 'bilinear'):

    # upsamples an image, using the desired interpolation method
    
    def up_sample(self, image, size, interp = 'bilinear'):
    
    # downsamples an image, using the desired padding strategy
    
    def down_sample(self, image, size, padding_type = 'mirror', 

    # Implements the down operation of the pyramid using the upsample function
    
    def down(self, level):

    # Implements the up operation of the pyramid using the downsample function
    
    def up(self, level):
    
    # builts the whole pyramid using up function
    
    def build(self):
    
    # returns the desired level of the pyramid
    
    def get(self, level):
    
    # plots each pyramid level
    
    def show(self, name = 'gauss_pyramid'):
\end{lstlisting}

We decided to use this architecture because it needs to compute each level of the pyramid just once, this is useful when function $get$ is implement recovering each level in $O(1)$, in the same way, this implementation is more transparent for an user, allowing easy and fast use of the pyramid as shown below. 

\begin{lstlisting}[language=python][h!]
pyramid = gaussian_pyramid(img, levels = 7)
pyramid.build()
\end{lstlisting}

Some details of the implementation includes: it can handle multichannel and grayscale images indifferently, it implements only bilinear interpolation\footnote{Based on \url{http://supercomputingblog.com/graphics/coding-bilinear-interpolation/}}, the $down$ function may not return an image of the size of the original one whenever the height or width of the image is odd.

The results of the applying the $up$ function to built the pyramid are shown in figure \ref{fig:gaussian_pyramid}

\begin{figure}[h!]
\centering
  \centering
  \includegraphics[width=0.9\linewidth]{output/gaussianPyramid.pdf}
  \caption{Gaussian Pyramid}
\label{fig:gaussian_pyramid}
\end{figure}

The result of the $down$ function comparing with the original pyramid level are shown in figure \ref{fig:down-gauss-results}. It can be seen that the upsampled image is not equal to the original one, there is a lost of information that is not recoverable.

\begin{figure}[!h]
\centering
\begin{subfigure}{0.5\textwidth}
  \centering
  \includegraphics[width=0.8\linewidth]{input/p1-1-3.png}
  \caption{Original image}
\end{subfigure}%
\begin{subfigure}{0.5\textwidth}
  \centering
  \includegraphics[width=0.8\linewidth]{output/upSample.jpg}
  \caption{Upsampled image}
\end{subfigure}%
 \caption{Comparison of results of $down$ function}
\label{fig:down-gauss-results}
\end{figure}

\section{Laplacian Pyramid}

Similar to Gaussian Pyramid, the implementation of the Laplacian Pyramid, follows the next model:

\begin{lstlisting}[language=python][h!]
class laplacian_pyramid:
    pyramid = []
    
    # Initialize the pyramid based on an image and the number of levels, the kernel is define
    # for the Gaussian pyramid used to build the Laplacian Pyramid
    
    def __init__(self, img, levels, kernel = None, gauss_kernel_par = 0.3):  
    
    # Addition or subtraction an image of level "i" in the 
    # pyramid with the upsampled image of the level i+1 
    
    def gauss_operation(self, gauss_cur, gauss_down, operation = '-'):

    # upsamples an image with the down function of the gaussian pyramid
    
    def up_sample(self, image, size):
    
    # upsamples an image of a level i of the laplacian pyramid and 
    # adds the image of level i-1 of the laplacian pyramid, obtaining 
    # the images of the level i of the gaussian pyramid
    
    def down(self, up_level_img, cur_level_img):
    
    # Implements the up operation of the pyramid using the downsample 
    # function, this function allows us create one level the 
    # laplacian pyramid
    
    def up(self, level, gaussian_pyramid):
    
    # builts the whole pyramid using up function
    
    def build(self):  
    
    # Reconstruct the original image using the laplacian pyramid
    
    def reconstruct(self):
    
    # returns the desired level of the pyramid
    
    def get(self, level):
    
    # plots each pyramid level
    
    def show(self, name = 'laplace_pyramid'):
      
\end{lstlisting}

As in the Gaussian implementation, we choose this architecture in order to compute each level of the pyramid just once, this be useful in the blending implementation.

Some details of the implementation includes: it can handle multichannel and grayscale images indifferently, the function  $gauss\_operation$ can handle the cases where the upsampled image does not have the same shape as the current level, the $down$ function up samples an image and sum it with another image, this function is the core the the $reconstruct$ function. The last level is the same in Laplacian pyramid and Gaussian pyramid.

The logic to reconstruct the image is shown in figure \ref{fig:reconstruction-laplace}. Initially, the last level $i$ of the Laplacian pyramid is upsampled, and then is added to the level $i-1$, the first row of figure \ref{fig:reconstruction-laplace} shows the elements inside the Laplacian pyramid, the second row shows the intermediate results of upsampling an image that was previously added with the corresponding Laplace pyramid level. Note that the results of the sum up (third row in figure \ref{fig:reconstruction-laplace}) are more clean and look like the original image.

\begin{figure}[h!]
\hspace{-1cm}
  \includegraphics[width=1.1\linewidth]{output/reconstructionImage.pdf}
  \caption{Reconstruction process}
\label{fig:reconstruction-laplace}
\end{figure}
