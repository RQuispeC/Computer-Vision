\section{Transform}

In order to explain our algorithm consider the following notation: $i$ refers to the $i-th$ frame in the original video, and $i'$ refers to the $i-th$ frame in the stabilized video.

Initially, we consider the following algorithm to compute $i'$:
\begin{enumerate}
	\item extract keypoints($X$) from $(i-1)'$
	\item extract keypoints($Y$) from $i$
	\item find matches($M$) between $X$ and $Y$
	\item compute transformation $T$ from $M$.
	\item apply transformation $T$ to $i$
\end{enumerate}

With this approach, we were obtaining a lot of misstransformations(Figure xxx) because the difference between $i$ and $(i-1)'$ are complex. Thus we defined anothe approach:   

\begin{enumerate}
	\item extract keypoints($X$) from $i-1$
	\item transform keypoints location of $i-1$ based on $(i-1)'$
	\item extract keypoints($Y$) from $i$
	\item find matches($M$) between $X$ and $Y$
	\item compute transformation $T$ from $M$.
	\item apply transformation $T$ to $i$
\end{enumerate}

We assume that the difference between adjacent frame in the original video is small. Thus, we compare the feature vector of $i$ and $i-1$, but the keypoints location of $i$ and $(i-1)'$.

Once we have the transformation of parameters we apply it for every pixel in $i$, depending of the movement, some black holes appears(Figure xxx). Thus, we interpolate this points with the average of four neighbors and fill with zero the cases when the neighbors are also empty.
